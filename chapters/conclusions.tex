\chapter{Conclusions}
\label{ch:conclusions}

\section{Summary}
\label{sec:summary}

\section{Reflections}
\label{sec:reflections}

\section{Future Work}\label{sec:futureWork}

If more time were available for this project the next step would be to tackle a more complex version of the game.
In this second iteration of the project limit texas hold'em would be tackled with the end goal of recreating
the results shown by Heinrich in his PhD thesis\citep{heinrich2017reinforcement}.
In this case we would measure win rate against other texas hold'em agents in order to evaluate
the success or failure of our product.

The final step of this project would be to attempt to extend the same methods to no-limit texas hold'em.
This game has a much larger state space(circa. $10^{164}$) than limit texas hold'em(circa. $10^{17}$).
This final approach would likely require the introduction of neural networks in order to provide accurate
generalisation between states in oder to reduce the complexity of the problem.