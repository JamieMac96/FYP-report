\chapter{Introduction (10)}
\label{ch:intro}
In this section I will introduce the subject area of this Final year Project (FYP).
I will then go on to give an overview of the report, establish some goals for the project along with some of
the motivations for choosing this subject area.


\section{Overview (3)}\label{sec:overview}
Since the inception of machine learning, games have been a key problem area that has seen a lot of focus from
top academics.
Furthermore, the development of some machine learning strategies that can be applied to games has also lead to these
strategies being applied in many different domains, many of which being very beneficial in practice.


\section{Objectives (2)}\label{sec:objectives}
\subsection{Reinforcement Learning for Leduc Hold'em}\label{subsec:primaryObjectives}
In the past, methods such as counterfactual regret minimization (CFR) have been used to develop agents that can
play no-limit texas hold'em to a superhuman level.
This includes the 2018 champion of the Annual Computer Poker Competition, slumbot\citep{jackson2013slumbot}.
There have also been attempts to solve the limit version of the game using reinforcement learning
(RL)\citep{heinrich2016deep}.

Throughout the course of this project I will be using an iterative approach to solving the problem.
As such the first agent that I will develop will seek to tackle a simplified version of hold'em.
Specifically I will be attempting utilise the algorithm outlined in\citep{heinrich2015fictitious} to develop an agent
that can play a simplified version of texas Hold'em called Leduc Hold'em.

\subsection{Apply the Algorithm to a More Complex Poker Variant}\label{subsec:pokerPlayingAgent}
When I have successfully applied the algorithm to Leduc Hold'em my goal is to then apply the same algorithm
to a more complex game.

\subsection{Web Interface to Facilitate Play Against the Agent}\label{subsec:webInterface}
The focus of this report will largely be research.
However it is also my goal to create a product that will be fun and useful for the general public.
As such another objective will be to create a website that will allow users to play heads-up against the final product.

\subsection{Understanding Reinforcement Learning}\label{subsec:understandingRL}
As this project is very specific and academic, one of the larger challenges will be to gain a strong knowledge
of the domain.
This means learning the history of RL, the types of problems that it has been used to solve and the specific details of
different RL algorithms.

\subsection{Understanding ML in Imperfect Information Games}\label{subsec:aiInImperfectinfo}
A successful project will require a high degree of knowledge from the broader domain of RL. However, it is also the case
that I must become closely familiar with the existing academic literature in the area of RL with respect to imperfect
information games.
This will allow me to avoid taking approaches that have previously shown to fail and also allow me to contribute to
the existing literature without simply replicating what has already been done.


\section{Contribution (1)}\label{sec:contribution}

\section{Methodology}\label{sec:methodology}

\section{Motivation (2)}\label{sec:Motivation}