\chapter{Background (18)}
\label{ch:background}

The aim of this chapter is to give the reader background information on the problem domain in order for them
to understand the rest of the report.
This will consist first of an introduction to machine learning.
Then we will go into more detail on the areas of reinforcement learning and texas hold'em.
Finally we will take a deeper dive on the literature surrounding how we can utilise reinforcement learning
to tackle the problem of texas hold'em.

\section{Introduction to Machine Learning (1) w5}\label{sec:introductionToMachineLearning}

Machine learning is an area of computer science that tackles how we construct computer programs that improve
with experience\cite{mitchell1997machine}.
The term was coined by Arthur Samuel in his 1959 paper where he discussed machine learning methods using
checkers as his problem area.
Since then there has been a great deal of advancement in the field.
Some of the notable early contributions being the discovery of recurrent neural networks in 1982 and
the advancement of reinforcement learning by the introduction of Q-Learning in 1989.
Recently we have seen some of this early academic work culminate in more practical achievements such as
Facebook's DeepFace system which, in 2014,  was shown to be able to recognise faces at a rate of 97.35\% accuracy,
a rate that is comparable to that of humans.
Another example of recent achievement is Google's AlphaGo program which, in 2016, became the first program to beat
a professional human player.

It should be becoming clear that machine learning can be a solution to a wide scope of problems and as
both hardware and software continue to improve this scope will only continue to widen.
We are starting to see machine learning systems become a key component of many companies business model.
Since certain machine learning techniques are great at prediction, machine learning has been widely used
for content discovery by companies such as Google and Pinterest.
Other business applications include the use of chatbots as a part of customer service, self-driving cars
and even in the field of medical diagnostics.

Since there is such a large range of actual and potential applications for machine learning it would be good for
us to understand how different methodologies can be applied to solve different types of problems.
In the next we will discuss just that.


\section{Machine Learning Categories (3)}\label{sec:mlCategories}

\subsubsection{Supervised Learning}
Supervised learning involves an agent which observes some example input-output pairs and learns
a function that maps from input to output.\cite{russell2016artificial}.
This learned function can then be used on new input data, that wasn't used to train the agent and the
agent should be able to give an accurate output.
As such this learning task is a generalization problem.
The agent must be able to identify general features of the input data and how they map to the output.



\subsubsection{Unsupervised Learning}
\subsubsection{Reinforcement Learning}

\section{History of Reinforcement Learning (1) w5}\label{sec:rlhistory}


\section{Applications of Reinforcement Learning (1) w5}\label{sec:rlApplications}


\section{Reinforcement Learning Methods (9) w6}\label{sec:rlMethods}

\subsection{Dynamic Programming (3)}\label{subsec:dp}

\subsection{Monte Carlo (3)}\label{subsec:mc}

\subsection{Temporal Difference Learning (3)}\label{subsec:td}


\section{Reinforcement Learning In Large State Spaces (1) w7}\label{sec:rlLargeStateSpace}