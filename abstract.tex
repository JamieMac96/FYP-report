%%
%% Author: jamie
%% 04/10/18
%%

\begin{abstract}
    This report investigates the possibility of applying reinforcement learning techniques
    to imperfect information games such as poker.
    A detailed research process is documented which begins with high-level reinforcement
    learning concepts and leads to specific algorithms utilised to solve games like poker.
    After review of the relevant literature, a modified version of Monte Carlo Tree Search was
    selected as the algorithm that would be utilised to solve our poker variant, Leduc Hold'em.
    The project set out to re-create the results of a recent Ph.D. thesis by Johannes Heinrich
    that demonstrated the effectiveness of this method.
    A number of experiments were conducted, utilising various metrics to measure the performance
    of the poker agent created.
    The results of these experiments were then recorded and analysed.
    A prototype game was created to allow users to interact with an instance of the trained agent.
    Monte Carlo Tree search was found to be a relatively effective method for strategy development in
    Leduc Hold'em agents with the agent performing well against human opponents and achieving
    an exploitability level of 2.2 in our final experiment.
\end{abstract}